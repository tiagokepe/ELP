\documentclass{beamer}
% Class options include: notes, notesonly, handout, trans,
%                        hidesubsections, shadesubsections,
%                        inrow, blue, red, grey, brown

% Theme for beamer presentation.
\usepackage{beamerthemesplit} 
%\usepackage{amsmath,amssymb}
\usepackage{url}
\usepackage[brazil]{babel}
\usepackage[latin1]{inputenc}


\title{Concorr�ncia em Ada}    % Enter your title between curly braces
\author{Antonio Carlos, Claudinei e Tiago}                 % Enter your name between curly braces
\institute{Departamento de Inform�tica \\ Universidade Federal do Paran�}      % Enter your institute name between curly braces
\date{\today}     

\begin{document}

\begin{frame}
  \titlepage
\end{frame}

\section[Outline]{}

% Creates table of contents slide incorporating
% all \section and \subsection commands
\begin{frame}
  \tableofcontents
\end{frame}

\section{Introdu��o}

\begin{frame}
 \frametitle{Introdu��o}
 	
	Ada suporta programa��o concorrente em n�vel de linguagem.
	\begin{itemize}
		\item Melhor utiliza��o do processador. Evita ociosidade durante opera��es de IO.
		\item Permitir que mais de um processador resolva o problema.
		\item Modelar paralelismo no mundo real. Por exemplo, aplica��es em tempo real e sistemas embarcados..
	\end{itemize}

\end{frame}
  
\section{Task}

\begin{frame}
 \frametitle{Task}
 
  Nome dado as atividades concorrentes em Ada.
  Cada \textit{Task} possui sua pr�pria \textit{thread} de controle.

\end{frame}

\subsection{Declara��o}
  \begin{frame}
     
    \frametitle{Declara��o}
      \begin{block}{}
	  	procedure blablabla
			task A;
			task B;
		end;
      \end{block}
      

  \end{frame}
  
  
\subsection{Comportamento}
  \begin{frame}
    \frametitle{Comportamento}
  \end{frame}



\end{document}
